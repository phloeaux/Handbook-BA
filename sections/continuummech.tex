\documentclass[../main.tex]{subfiles}
\graphicspath{{\subfix{../IMAGES/}}}


\begin{document}
\localtableofcontents
\subsection{A body}
\quad \underline{Eulerian description of space :}\\
\begin{equation}
    \underline{x} = f(\underline{X}, t)
\end{equation}
$f$ is continuous and single valued. Here the vector x isn't attached to a particle X.\\

\quad \underline{Lagrangian description of space :}\\
\begin{equation}
    \underline{X} = \underline{x}(\underline{X}, t=t_0)
\end{equation}
Here if the particle X moves, the x vector follows it.\\

\quad \underline{Deborah's number :} \\
\begin{equation}
    De = \frac{\tau}{t_m}
\end{equation}
With $\tau$ the observational timescale and $t_m$ the material relaxation timescale. It describes how a solid reacts to forces.\\

\subsubsection{Transformation of vector components and law}
$\underline{a} = \sum_{i=1}^3 a_i \hat{e_i} = a_i \hat{e_i} = a_j' \hat{e_j'}$\\
$a_j' = Q_{ji} a_i$, $Q_{ij} = \hat{e_i'}\cdot \hat{e_j}$\\
\begin{equation}
    [\underline{\underline{Q}}] = \begin{pmatrix}
        \hat{e_1'}\cdot \hat{e_1} & \hat{e_1'}\cdot \hat{e_2} & \hat{e_1'}\cdot \hat{e_3}\\
        \hat{e_2'}\cdot \hat{e_1} & \hat{e_2'}\cdot \hat{e_2} & \hat{e_2'}\cdot \hat{e_3}\\
        \hat{e_3'}\cdot \hat{e_1} & \hat{e_3'}\cdot \hat{e_2} & \hat{e_3'}\cdot \hat{e_3}\\
    \end{pmatrix}
\end{equation}


\subsubsection{Mass conservation}
We use here a right tetrahedron, with the area of oblique face being $\Delta \Sigma$.\\
Over a time $\Delta t$ a certain amount of mass $\Delta Q$ passes through this surface. \\
The density flux of continuum :\\
\begin{equation}
    j_n = \lim_{\substack{\Delta \Sigma \to 0\\ \Delta t \to 0 } } \frac{\Delta Q}{\Delta t \Delta \Sigma}
\end{equation}
The total mass being : $\int_{\Omega} \rho(\underline{x}, t)dV$.\\
The mass varies as the body moves : $(\frac{\partial}{\partial t} \int_{\Omega} \rho(\underline{x}, t)dV)dt$\\
It can vary in two ways :\\
\begin{enumerate}
    \item $-dt \int_{\partial \Omega} j_n d\Sigma'$\\
    \item $dt \int_{\Omega} Q(\underline{x},t)dV$ where Q is i.e. a chemical/nuclear reaction\\
\end{enumerate} 
\begin{equation}
    \frac{\partial}{\partial t} \int_{\Omega} \rho(\underline{x},t)dV = -\int_{\partial \Omega} j_n d\Sigma + \int_{\Omega}Q(\underline{x},t)dV
\end{equation}

If we let $\Omega \rightarrow 0$ then we get :\\
$\int_{\partial \Omega} j_n d\Sigma = j_1n_1d\Sigma + j_2n_2d\Sigma + j_3n_3 d\Sigma - j_nd\Sigma = (j_1n_1 + j_2n_2+j_3n_3 -j_n)d\Sigma \simeq 0$\\
\begin{equation}
    j_n = j_in_i
\end{equation}
With $n_i$ components of $\hat{\underline{n}}$. Thus, we have $j_i n_i = j_m' n_m'$ with the relation $n_i = n_m' Q_{mi}$\\
Therefore : $j_m' = j_i Q_{mi}$\\
The velocity is defined by :\\
\begin{equation}
    \underline{u} = \frac{\underline{j}}{\rho} = \frac{\partial \underline{x}}{\partial t} \Rightarrow \underline{j} = \rho \underline{u}
\end{equation}

We also now have the relation : $\int_{\partial \Omega} j_nd\Sigma = \int_{\partial \Omega} \underline{j}\cdot \underline{n}d\Sigma = \int_{\Omega} \underline{\nabla}\cdot \underline{j}dV$.\\

The mass conservation now becomes :\\
\begin{equation}
    \int_{\Omega}(\frac{\partial}{\partial t} \rho + \underline{\nabla}\cdot \underline{j} -Q)dV = 0
\end{equation}
The locality principle is now :\\
\begin{equation}
    \frac{\partial}{\partial t} \rho + \underline{\nabla}\cdot \underline{j} -Q = 0
\end{equation}

\subsection{Momentum conservation in continuum}
Consider a sub-volume $\Omega$ within a continuum body referenced to the laboratory coordinate frame such that the continuum body can flow through it.\\
The moment of the region $\Omega$ at time t is :\\
\begin{equation}
    \int_\Omega \rho \underline{u}dV
\end{equation}
The variation of the momentum of $\Omega$ in time interval dt is :\\
\begin{equation}
    (\frac{\partial}{\partial t} \int_\Omega \rho \underline{u} dV)dt
\end{equation}

The moment of region $\Omega$ varies in two ways :\\
\begin{itemize}
    \item momentum can be injected from a mass force : $(\int_\Omega \rho \underline{F}dV)dt$ (F is an acceleration)\\
    \item via influx of momentum through $\partial \Omega$ : $-(\int_{\partial\Omega} (\rho \underline{u} \cdot \underline{n})\underline{u}d\Sigma)dt$\\
\end{itemize}
We also have $(\rho \underline{u} \cdot \underline{n}) \underline{u} = \rho u_i u_j n_j = \rho \underline{u} \underline{u} \underline{n}$\\

If we perform cuts in the volume, we can define a stress vector $\underline{t_n} = \lim_{dA\to 0} \frac{\mathcal{F}}{dA}$\\

\color{gray} $\underline{t_n}$ is a vector that has three components for the cut that we selected.\color{black}\\

By doing so along three axis, we need a new object that has the following properties :\\
\begin{itemize}
    \item independent of coordinate system (obey objectivity)\\
    \item map vectors (normal vectors) onto other vectors (stress vectors)\\
\end{itemize}

Therefore, we get the \textbf{stress tensor}. It codifies objectivity with a transformation law :\\
\begin{equation}
\begin{split}
    \underline{\underline{\sigma}}\\
    \sigma_{kl}' = Q_{ki} Q_{lj} \sigma_{ij}
\end{split}
\end{equation}
Each component of $\underline{\underline{\sigma}}$ has unit of stress.\\

Thus, the momentum injected into $\Omega$ in time interval dt is :\\
\begin{equation}
    -(\int_{\partial \Omega} (\rho \underline{u} \underline{u}-\underline{\underline{\sigma}})\underline{n}d\Sigma)dt
\end{equation}

Bringing back everything together we get :\\
\begin{equation}
    \frac{\partial}{\partial t} \int_\Omega \rho \underline{u} dV = \int_\Omega \rho \underline{F}dV- \int_{\partial \Omega} (\rho \underline{u} \underline{u}-\underline{\underline{\sigma}})\underline{n}d\Sigma
\end{equation}

Applying the divergence theorem, we get :\\
\begin{equation}
    \int_\Omega \frac{\partial}{\partial t} (\rho \underline{u}) + \underline{\nabla}\cdot (\rho \underline{u} \underline{u}-\underline{\underline{\sigma}}) - \rho \underline{F} dV= 0
\end{equation}

Because the volume is arbitrary, we get the local expression :\\
\begin{equation}
    \frac{\partial}{\partial t} (\rho \underline{u}) + \underline{\nabla}\cdot (\rho \underline{u} \underline{u}-\underline{\underline{\sigma}}) - \rho \underline{F} =0
\end{equation}

For a Cartesian coordinate system we can write : $\frac{\partial}{\partial t}(\rho u_i) + \frac{\partial}{\partial x_j}(\rho u_j u_j - \sigma_{ji}) = \rho F_i$\\

Or even : \\
\begin{equation}
    \frac{\partial \underline{u}}{\partial t} + (\underline{u}\cdot \underline{\nabla})\underline{u} = \underline{F} + \frac{1}{\rho} \underline{\nabla}\cdot \underline{\underline{\sigma}}
\end{equation}

We define the acceleration term as :\\
$(\frac{\partial }{\partial t} + (\underline{u}\cdot \underline{\nabla}) \underline{u})$. With the term in between the parentheses the material derivative ($\frac{d}{dt}$)\\

Thus, we get the local equation for linear momentum : \\
\begin{equation}
    \frac{d \underline{u}}{dt} = \underline{F} + \frac{1}{\rho} \underline{\nabla}\cdot \underline{\underline{\sigma}}
\end{equation}


From angular momentum balance, we get $\int_\Omega \sigma_{ij}\sigma_{ji}dV = 0 \Leftrightarrow \sigma_{ij} = \sigma_{ji}$\\

\subsection{Indicial notation}
We have $\delta_{ii} = 3$, $\delta_{im} T_{mj} = T_{ij}$, $\hat{e_i} \cdot \hat{e_j} = \delta_{ij}$\\

\subsubsection{Levi-civita symbol}
$\varepsilon_{ijk} = \begin{cases}+1 & \text{if ijk form an even permutation of 1,2,3} \\
-1 & \text{if it forms and odd one}\\
0 & \text{else}\\
\end{cases}$\\

\quad \underline{cross product}\\
$\Vec{a}\times \Vec{b} = (a_i \hat{e_i}) \times (b_j \hat{e_j}) = a_ib_j (\hat{e_i} \times \hat{e_j}) = a_ib_j \varepsilon_{ijk} \hat{e_k}$\\

\subsection{Stress vector/tensor components}
We define two types of components in the stress tensor : \\
\begin{itemize}
    \item normal component : along the normal of the face\\
    \item shear component : in the plane of the face\\
\end{itemize}

We also define the $\underline{t}_{\hat{n}}$ components by : \\
\begin{equation}
    \underline{t}_{\hat{n}} = \hat{n} \cdot \underline{\underline{\sigma}} = \sigma_{nj}\hat{e}_j
\end{equation}
In most circumstances, $\underline{\underline{\sigma}}$ is symmetric : $\underline{\underline{\sigma}} \underline{n} = \underline{n}\cdot \underline{\underline{\sigma}}$\\

\subsubsection{Transformation of tensor components}
\begin{equation}
    \lambda_{ij} = \hat{e}_i' \cdot \hat{e}_j \Leftrightarrow \underline{\underline{\lambda}} = \hat{e}_i' \hat{e}_j = \hat{e}_i' \otimes \hat{e}_j
\end{equation}

Therefore, we have $\hat{e}_i' = \lambda_{ij} \hat{e}_j$ and $\hat{e}_i = \lambda_{ji} \hat{e}_j'$\\

We can also define $\lambda_{ij} \lambda_{kj} = \delta_{ik}$\\

\begin{table}[hbt!]
    \centering
    \begin{tabular}{c|c}
        Rank/order & Transformation law \\
        \hline
        0 (scalar) & $\alpha'=\alpha$\\
        1 (vector) & $a_i' = \lambda_{ij} a_j$\\
        2 (tensor) & $T_{ij}' = \lambda_{ik} \lambda_{jl} T_{kl}$\\
        3 & $S_{ijk}' = \lambda_{in} \lambda_{jm} \lambda_{kl} S_{nml}$\\
        4 & $C_{ijkl}' = \lambda_{in} \lambda_{jm} \lambda_{kr} \lambda_{ls} C_{nmrs}$\\
    \end{tabular}
    %\caption{Transformation laws}
\end{table}
\color{gray}Some helpful rules : \begin{itemize}
    \item addition : the sum of two tensors is a tensor\\
    \item multiplication : we can define a variety of different products : $a_i a_j$, $a_i a_j a_k$, $T_{ij} = T_{kl}$\\
    \item quotient : if $a_i$ and $T_{ij}$ are vector and tensor components, and $a_i = T_{ij} b_j$ then $b_j$ are components of a vector\\
\end{itemize} \color{black}

\subsubsection{Principle stresses and characteristics equation for stress tensor}
The transformation laws and symmetry of $\underline{\underline{\sigma}}$ ensure that \textbf{3 perpendicular directions} along which there are no shear stresses can be found.\\
In matrix form, we can find a basis where $[\underline{\underline{\sigma}}]$ is diagonal. These directions are \textbf{eigenvectors} and the principle stresses are \textbf{eigenvalues}.\\
If we order $\sigma_I \leq \sigma_{II} \leq \sigma_{III}$ then for a face oriented along $\hat{n}$ : \\
\begin{equation}
    \sigma_I \leq \sigma_{\hat{n}} \leq \sigma_{III} \Leftrightarrow \sigma_{\hat{n}} = \hat{n}\cdot \underline{t}_{\hat{n}}
\end{equation}

The eigenvalues are defined by : $\sigma_{ij}n_j- \sigma n_i = 0 $\\
If $\underline{\underline{\sigma}}$ is real valued and symmetric then :\\
the determinant of the coefficient matrix should vanish :\\
\begin{equation}
    \det \begin{bmatrix}
    \sigma_{11}-\sigma & \sigma_{12} & \sigma_{13}\\
    \sigma_{21} &\sigma_{22}-\sigma & \sigma_{23}\\
    \sigma_{31} & \sigma_{32} & \sigma_{33}-\sigma\\
    \end{bmatrix} = -\sigma^3 + I_1 \sigma^2 + I_2 \sigma + I_3 = 0
\end{equation}
With $I_1 = \sigma_{ii}$ the trace\\
$I_2 = \frac{1}{2}(\sigma_{ij}\sigma_{ji} - (I_1)^2)$\\
$I_3 = \det(\underline{\underline{\sigma}})$\\

$\mathbf{I_1, I_2, I_3}$ \textbf{are invariant}.\\

\subsection{Fields}
\subsubsection{Scalar field and its gradient}
\quad For a scalar valued function of a vector, $\phi(\Vec{r})$, associated with $\phi$ is a vector field formed by calculating the gradient of $\phi$.\\
$\Vec{\nabla}\phi = \Vec{\nabla} \phi \cdot d\Vec{r} = \phi(\Vec{r}+d\Vec{r}) - \phi(\Vec{r})$, let $dr = \lvert d\Vec{r}\rvert$, $\hat{n}$ the unit vector along $d\Vec{r}$ such that $\hat{n} = \frac{d\Vec{r}}{dr}$ : $\frac{d\phi}{dr} = \Vec{\nabla} \phi \cdot \hat{n}$\\
The component of $\Vec{\nabla} \phi$ in a direction $\hat{e_i}$ gives the rate of change of $\phi$ along $\hat{e_i}$. With indicial notation :\\
$\frac{\partial \phi}{\partial x_i} = \frac{\partial \phi}{\partial r}\rvert_{\hat{e_i}} = \Vec{\nabla}\phi \cdot \hat{e_i}$, or $\Vec{\nabla} \phi = \phi_{,i} \hat{e_i}$\\

\color{gray}\underline{Geometric interpretation :} suppose $\phi$ represents the temperature along a surface of constant temperature : $\phi =$constant. If $\Vec{r}$ is a point on this isothermal surface, all points along the surface at $\Vec{r}+d\Vec{r}$ corresponds to $d\phi=0$ $\Rightarrow \Vec{\nabla}\phi \cdot d\Vec{r} = 0$ on isosurface $\Rightarrow \Vec{\phi}$ is a vector perpendicular to isosurface at point $\Vec{r}$. For any isosurface, the maximum gradient is normal to it.\color{black}\\

 \subsubsection{Gradient of vector field}
\quad Given $\Vec{v}$ a vector valued function of space : $\Vec{v}(\Vec{r})$. The gradient of $\Vec{v}(\Vec{r})$ : $\Vec{\nabla} \Vec{v}$ is a second order tensor that, when operating on a differential space vector, $d\Vec{r}$ : \\
\begin{equation}
    d\Vec{v} = \Vec{v}(\Vec{r}+d\Vec{r})-\Vec{v}(\Vec{r}) = (\Vec{\nabla}\Vec{v})d\Vec{r}
\end{equation}
In Cartesian coordinates, we get $\frac{d\Vec{v}}{dr}\rvert_{\hat{n}} = \frac{\partial\Vec{v}}{\partial x_j} = (\Vec{\nabla} \Vec{v})\hat{e_j}$ or : $(\Vec{\nabla}\Vec{v})_{ij} = \frac{\partial v_i}{x_j} = v_{i,j}$\\
\begin{equation}
    \Vec{\nabla} \Vec{v} = \frac{\partial v_i}{\partial x_j} \hat{e_i}\otimes \hat{e_j}
\end{equation}

\subsubsection{Divergence and its theorem}
\quad \underline{Definition of divergence :} \\
\begin{equation}
    \text{div} \Vec{F} = \lim_{\Omega \rightarrow 0} \frac{1}{\Omega} \int_{\partial \Omega} \Vec{F}\cdot \hat{n}d\Sigma
\end{equation}
We therefore get : \\
\begin{equation}
    \text{div}\Vec{F} = \frac{\partial F_1}{\partial x_1} + \frac{\partial F_2}{\partial x_2} + \frac{\partial F_3}{\partial x_3} = \Vec{\nabla} \cdot \Vec{F} = F_{i,i}
\end{equation}

For a tensor, we get the divergence theorem : \\
\begin{equation}
    \int_{\partial \Omega_i}\hat{n} \cdot \underline{\underline{\sigma}} d\Sigma \simeq \sum_{i=1}^N (\Vec{\nabla}\cdot \underline{\underline{\sigma}})_i \Omega_i = \int_{\Omega} \Vec{\nabla} \cdot \underline{\underline{\sigma}} d\Omega \quad \text{Assuming, $\Omega_i\rightarrow 0$, $N\rightarrow \infty$} 
\end{equation}
 
\quad \underline{Divergence of a tensor :} We define the divergence of a tensor as :\\
\begin{equation}
    \Vec{\nabla} \cdot \underline{\underline{T}} = T_{ji,j} \hat{e_i}
\end{equation}
Also : $(\Vec{\nabla} \cdot \underline{\underline{T}}) \cdot \Vec{a} = \Vec{\nabla} (\Vec{a} \cdot \underline{\underline{T}}^T) = (a_i T_{ji})_{,j}$\\
And : $(\Vec{\nabla} \cdot \underline{\underline{T}})_i \hat{e_i} \cdot a_k \hat{e_k} = (\Vec{\nabla} \cdot \underline{\underline{T}})_i \delta_{ik} a_k = (\Vec{\nabla} \cdot \underline{\underline{T}})_i a_i = T_{ji,j}a_i$\\

\subsection{Kinematics of continua}
\subsubsection{Time-dependant motion}

A particle's path is defined with respect to a given basis $\hat{e}_i$ : \\
\begin{equation}
    \underline{r}(t) = x_1(t) \hat{e}_1 + x_2(t) \hat{e}_2 + x_3(t) \hat{e}_3
\end{equation}
For a given collection of N particles, we can define the path for each of them : $\underline{r}_n = \underline{r}_n(t)$ for $n=1,\dots, N$.\\
The motion of a continuum can be described in the limit $N\rightarrow \infty$.\\

In a continuous body, the particle of interest is identified by its instantaneous coordinates : $x_1, x_2,x_3$.\\
By extension, the path-lines for each particles are described by the equation : \\
\begin{equation}
    \underline{x} = \underline{x}(\underline{X},t)
\end{equation}
Where $\underline{X} = \underline{x}(\underline{X},t_0)$.\\

Here, $\underline{X}$ denotes the material coordinates of each particle in the continua and $x_i(\underline{X},t)$ defines the path lines for each particles $X_i$.\\

\subsubsection{Material, spatial description of continua, transport in continua}
We know that quantities such as temperature ($\theta$), velocity ($\underline{v}$), stress ($\underline{\underline{\sigma}}$), can change in time. However, when a continuum is in motion, there are two main ways to describe changes in these quantities over time : \\
\begin{enumerate}
    \item Material or Lagrangian description : changes are calculated with respect to the bodie's material coordinates. We associate these quantities with the material particle\\
    \item Spatial or Eularian description : changes are calculated with respect to the laboratory coordinates.\\
\end{enumerate}

The time rate of change of some property of a material particle, $\theta, \underline{v}, \underline{\underline{\sigma}}$, is given by its material derivative. This is denoted by $\frac{d}{dt}$ or $\frac{D}{Dt}$ : \\
\begin{enumerate}
    \item In the material description : $\frac{d \theta}{dt} = \frac{\partial \theta}{\partial t}$\\
    \item In the spatial description : $\frac{d\theta}{dt} = \frac{\partial \theta}{\partial t} + \frac{\partial \theta}{\partial x_i} \frac{\partial x_i}{\partial t}$, for Cartesian coordinates we have :\begin{equation}
        \frac{d\theta}{dt} = \frac{\partial \theta}{\partial t} + (\underline{v} \cdot \underline{\nabla}) \theta
    \end{equation}
\end{enumerate}
For a material particle, we are interested to know its acceleration. To calculate a particle's acceleration, we evaluate the material derivative of the particle's velocity.\\
\begin{equation}
    \underline{a} = \frac{d \underline{v}}{dt} = \frac{\partial \underline{v}}{\partial t} + (\underline{v} \cdot \underline{\nabla}) \underline{v} 
\end{equation}

\subsubsection{Displacement and deformation}
The vector connecting a particle at reference position $\underline{X}$ to its current location $\underline{x}$ is called the \textbf{displacement vector} $\underline{u}(\underline{X},t)$\\

Once the path-lines $\underline{x}(\underline{X},t)$ are known, we also know the displacement vector field \begin{equation}
    \underline{u} (\underline{X},t) = \underline{x}(\underline{X},t)-\underline{X}
\end{equation}

The deformation of the gradient tensor is therefore defined as :\\
\begin{equation}
    \underline{\underline{F}} = \underline{\underline{I}} + \underline{\nabla} \underline{u}
\end{equation}
With the relation $d\underline{x} = \underline{\underline{F}} d\underline{X}$\\

Also, we have $dS = \lvert d\underline{X}\rvert $, $ds = \lvert d\underline{x}\rvert$\\
The \textbf{right Cauchy-Green deformation tensor} is defined as :\begin{equation}
    \underline{\underline{C}} = \underline{\underline{F}}^T\underline{\underline{F}}
\end{equation}

\quad \underline{Stretch :} \begin{equation}
    \frac{\lvert d\underline{x}\rvert}{\lvert d\underline{X}\rvert} = \lambda
\end{equation}

Pure translation : $\underline{\underline{F}}$ is the identity matrix\\
Pure rotation : $\lvert \underline{\underline{F}} \rvert = 1$\\
Pure shear deformation : $\underline{\underline{F}}$ isn't symmetric\\

\quad \underline{General planar deformation :}\\
We can rewrite (stretch plus rotation) :
\begin{equation}
\underline{\underline{F}} = \underline{\underline{R}} \underline{\underline{U}}
\end{equation}
With $\underline{\underline{R}}$ the rotation and $\underline{\underline{U}}$ the stretches.\\

We have $\underline{\underline{U}}$ diagonal with its coefficients $\lambda_i = \frac{\lvert d\underline{x}_i\rvert}{\lvert d\underline{X}_i\rvert}$ With $d\underline{x}_i =\underline{\underline{F}} d\underline{X}_i $\\

Therefore, $\underline{\underline{R}} = \underline{\underline{F}}\underline{\underline{U}}^{-1}$


Also, we can have a rotation plus stretch :\\
\begin{equation}
    \underline{\underline{F}} = \underline{\underline{V}} \underline{\underline{R}}
\end{equation}
With $\underline{\underline{R}}$ the same matrix. And $\underline{\underline{V}} = \underline{\underline{F}} \underline{\underline{R}}^{-1} = \underline{\underline{F}} \underline{\underline{R}}^T$\\

\warning As $\underline{\underline{R}}$ is a rotation tensor, we have that $\underline{\underline{R}}^{-1} = \underline{\underline{R}}^T$. Also, $\underline{\underline{U}}$ is symmetric.\\

\subsubsection{Infinitesimal strain tensor}
We have by definition : $\underline{\underline{C}} = \underline{\underline{I}} + \underline{\nabla}\underline{u} + (\underline{\nabla}\underline{u})^T+ (\underline{\nabla} \underline{u})^T(\underline{\nabla} \underline{u})$\\
We therefore define the \textbf{Lagrange strain tensor } \begin{equation}
    \underline{\underline{E}}^* = \frac{1}{2} (\underline{\nabla}\underline{u} + (\underline{\nabla}\underline{u})^T+ (\underline{\nabla} \underline{u})^T(\underline{\nabla} \underline{u}))
\end{equation}

For infinitesimal deformation : \begin{equation}
    \underline{\underline{E}} = \frac{1}{2}(\underline{\nabla} \underline{u} + (\underline{\nabla} \underline{u})^T)
\end{equation}
The infinitesimal strain tensor.\\
In Cartesian coordinates : $E_{ij} = \frac{1}{2}(\frac{\partial u_i}{\partial X_j} + \frac{\partial u_j}{\partial X_i})$\\

\subsubsection{Principle of strain}
$\underline{\underline{E}}$ is symmetric with real components. Therefore, we have three mutually perpendicular eigenvectors and we can diagonalise E.\\
The geometric interpretation of \textbf{eigenvectors}, $\{\hat{n}_i\}$, direction along which the deformation is purely elongation/contraction.\\
The \textbf{eigenvalues} are the principle strains $E_i$\\
The principle strains contain the maximum and minimum strain along any direction.\\
Given $\underline{\underline{E}}$, $E_i$ are calculated as the roots of the characteristic polynomial : $\lambda^3-I_1 \lambda^2+I_2 \lambda - I_3 = 0$. \\
The invariants of $\underline{\underline{E}}$ are therefore :\begin{itemize}
    \item $I_1 = E_{ii}$\\
    \item $I_2 = \begin{bmatrix}
        E_{11} & E_{12}\\
        E_{21} & E_{22}\\
    \end{bmatrix} + \begin{bmatrix}
         E_{22} & E_{23}\\
        E_{32} & E_{33}\\
    \end{bmatrix} + \begin{bmatrix}
        E_{11} & E_{13}\\
        E_{31} & E_{33}\\
    \end{bmatrix}$\\
    \item $I_3 = \det (\underline{\underline{E}})$\\
\end{itemize}

\quad \underline{Dilatation :} allows us to interpret $E_{ii}$ geometrically by considering three material lines, emanating from a point P in the continuum along $\underline{\underline{E}}$'s eigenvectors.\\

The change in volume is given by $\Delta (dV) = dS_1 dS_2 dS_3(1+E_1)(1+E_2)(1+E_3)-dS_1dS_2dS_3 \simeq dS_1dS_2dS_3(E_1+E_2+E_3)$\\

The unit volumetric change : \begin{equation}
    e = \frac{\Delta (dV)}{dV} = E_{ii}
\end{equation}
This is defined as the \textbf{dilatation}.\\

\quad \underline{Geometric interpretation of $\underline{\underline{E}}$'s components :}\\
Suppose we have a material vector, $d\underline{X} = dS \hat{n}$, $\hat{n}$ is a unit vector along $d\underline{X}$. $dS$ is the magnitude of $d\underline{X}$. After deformation, we get : \\
$ds^2 = dS^2+2dS^2(\hat{n}\cdot \underline{\underline{E}} \hat{n})$ $\Rightarrow \frac{ds-dS}{dS} = \hat{n}\cdot \underline{\underline{E}} \hat{n} = E_{nn}$(no sum)\\

\underline{Diagonal :} elements of $\underline{\underline{E}}$ represents unit elongation on their respective coordinates direction \\

\underline{Off-diagonal :} components of shear. consider two material vector $d\underline{X}^{(1)} = dS_1 \hat{m}$ and $d\underline{X}^{(2)} = dS_2 \hat{n}$, m and n are orthogonal.\\
\begin{equation}
    \gamma = 2(\underline{m}\cdot \underline{\underline{E}}\underline{n})
\end{equation}

\subsection{Constitutive laws and hookean solid}
We will focus on linearly elastic material such that : \begin{itemize}
    \item deform slightly upon application of stress\\
    \item resume their under-formed state when stress is removed\\
\end{itemize}

To properly normalize, we define the stress as the load divided by the initial cross section area : $\sigma = \frac{P}{A_0}$ and the strain as : $\frac{\Delta l}{l}$\\
We also define Young's modulus as $E_y$\\

\quad \underline{Poisson ratio :} $\varepsilon_d = \frac{\Delta d}{d}$ the diametric strain. We have that $\frac{\varepsilon_d}{\varepsilon_a} = C$ a constant which is Poisson's ratio\\

For elasticity theory, we will consider material that are isotropic and homogeneous.\\

An applied torque/torsion generates angular displacement.\\

Therefore, the shear modulus is defined as : $\mu = \frac{M_t l}{I_p \theta} \Rightarrow I_p =  \frac{\pi r^4}{2}$ the second area moment.\\

We also have other constitutive response : \begin{itemize}
    \item ideal gaz law : PV=nkT\\
    \item Newtonian fluid : $\tau \propto \dot{\gamma} \Rightarrow \tau = \eta \dot{\gamma}$ \color{gray}Note : shear thickening : $\tau \propto \dot{\gamma}^n$,n>1\\
    shear thinning : $\tau \propto \dot{\gamma}^n$, n<1\color{black}\\
    \item Yielding behaviour\\
\end{itemize}

\subsubsection{Linear elastic solid}
Common features of our though experiment : \begin{enumerate}
    \item linear relation between applied loading and deformation\\
    \item loading rate does not change the measurements\\
    \item deformation go away when the force is released\\
    \item deformations are small\\
\end{enumerate}

\quad \underline{Stress function of strain/elasticity tensor :}\\
The cauchy stress tensor is : $\underline{\underline{\sigma}} = \underline{\underline{\sigma}}(\underline{\underline{\varepsilon}})$\\

\begin{equation}
    \sigma_{ij} = C_{ijkl} \varepsilon_{kl}
\end{equation}

$\underline{\underline{\underline{\underline{C}}}}$ obeys tensor transformation laws : =$C_{ijkl}' = \lambda_{im} \lambda_{jn} \lambda_{kr} \lambda_{ls} C_{mnrs}$\\
This 4-th order tensor has 81 unique components.\\

\quad \underline{Symmetry of stress strain tensor :}\\

We have $\sigma_{ij} = \sigma_{ji}$ and $\varepsilon_{kl} = \varepsilon_{lk}$ Therefore, $C_{ijkl} = C_{jikl} = C_{ijlk}$ which means that the number of independent components reduced to 36.\\

Now, we use the strain energy density function $W(\varepsilon_{ij})$ such that $\sigma_{ij} = \frac{\partial W}{\partial \varepsilon_{ij}}$ this implies $C_{ijkl} = C_{klij}$ which again reduces the number of independent components to 21.\\

\quad \underline{Isotropic solids and the Lamé coefficients :}\\
For isotropic medium elasticity tensor cannot depend on coordinate direction : $C_{ijkl}' = C_{ijkl}$ for all orthogonal bases\\

The only isotropic 2D- tensor is : $\underline{\underline{I}}$\\
For a 4-th order tensor, there are 3 isotropic tensors : \begin{equation}
    A_{ijkl} = \delta_{ij} \delta_{kl}, B_{ijkl} = \delta_{ik} \delta_{jl}, H_{ijkl} = \delta_{il} \delta_{jk}
\end{equation}
We can therefore reduce the number of components of $\underline{\underline{\underline{\underline{C}}}}$ to two independent components for linear, homogeneous, isotropic solid.\\
\begin{equation}
    C_{ijkl} = \lambda A_{ijkl} + \alpha B_{ijkl} + \beta H_{ijkl}
\end{equation}
We can now define Hooke's law as :\\
\begin{equation}
    \sigma_{ij} = C_{ijkl} \varepsilon_{kl} = \lambda \delta_{ij} \varepsilon_{kk} + 2\mu \varepsilon_{ij} \Rightarrow \underline{\underline{\sigma}} = \lambda e \underline{\underline{I}}+2 \mu \underline{\underline{\varepsilon}} 
\end{equation}
Where $e$ is the dilatation $e = \varepsilon_{kk}$ and $\lambda$ and $\mu$ are Lamé coefficients\\

\quad \underline{$\underline{\underline{\varepsilon}}(\underline{\underline{\sigma}})$ :}\\
\begin{equation}
    \varepsilon_{ij} = \frac{1}{2\mu} [\sigma_{ij} - \frac{\lambda}{3\lambda + 2\mu} \sigma_{kk} \delta_{ij}]
\end{equation}

From that, we can extract : $e = \frac{1}{3\lambda + 2\mu} \sigma_{kk}$\\

Suppose we have a uniaxial stress state : $\varepsilon_{11} = \frac{\lambda+\mu}{\mu(3\lambda+2\mu)}\sigma_{11}$ which leads to : \begin{equation}
    \frac{1}{E_y} = \frac{\lambda+\mu}{\mu(3\lambda + 2\mu)}
\end{equation}

We also have : $\varepsilon_{22} = \varepsilon_{33} = -\frac{\lambda}{2(\lambda+\mu)} \varepsilon_{11}$. This in turn leads to :\\
\begin{equation}
    \nu = \frac{\lambda}{2(\lambda+\mu)}
\end{equation}

\quad \underline{Simple shear and shear modulus :}\\
Taking a sheet of material and displacing one side of it such that only one pair of shear stresses is non zero, we get :\\
The \textbf{shear modulus} : $G = \frac{\tau}{2 \varepsilon_{12}}$\\

The lamé coefficient $\mu$ is the shear modulus. \\

If we consider a hydro-static state of stess : $\sigma = p \underline{\underline{I}}$. The deformation geometry will be essentially dilatation : $e = \frac{1}{3\lambda + 2\mu} \sigma_{kk}$\\
The bulk modulus is therefore defined as : \begin{equation}
    K = \frac{3\lambda+2\mu}{3}
\end{equation}

\quad \underline{Incompressible solid :}\\
An incompressible fluid does not change its volume under any stress state. An isotropic linearly elastic solid that is also incompressible with a Young's modulus $E_y$ as the following properties :\begin{itemize}
    \item $\nu = \frac{1}{2}$\\
    \item $\mu = \frac{E_y}{3}$\\
    \item $K\rightarrow \infty$, $\lambda \rightarrow \infty$ and $K-\lambda = \frac{2\mu}{3}$\\
\end{itemize}

\subsubsection{Introduction to boundary value problems}
\quad \underline{Governing equation (linear moment balance) :}\\
\begin{equation}
    \rho \underline{b} + \underline{\nabla}\cdot \underline{\underline{\sigma}} = \rho \underline{a}
\end{equation}

The structure of this equation is vectorial (set of 3 equations), describes 9 to 13 variables and is not closed.\\

A lot of structural application are static ($\rho \underline{a} = 0$). The structure can be efficient where the self weight of the structure is small compared with typical stresses ($\rho \underline{b} = 0$).\\
This leaves us with the \textbf{equilibrium equation} : \begin{equation}
    \underline{\nabla} \cdot \underline{\underline{\sigma}} = 0
\end{equation}
This problem is still not closed (3equations for 6 unknowns). In order to solve problems, we therefore need the appropriate boundary data.\\
Any function that satisfies the following 3 criteria is a solution : \begin{itemize}
    \item satisfies the governing equations : $\underline{\nabla} \cdot \underline{\underline{\sigma}}$\\
    \item satisfies any boundary conditions\\
    \item must have a compatible strain field\\
\end{itemize}

\quad \underline{Typical BC :}\\
\begin{itemize}
    \item stress BC : prescribed traction vector over a portion of the surface\\
    \item displacement BC : the displacement is prescribed along a portion of the body's boundary\\
\end{itemize}

\warning We cannot simultaneously describe a stress BC and displacement BC along the same portion of the body's boundary\\

\quad \underline{Compatibility :}\\
The relation between strain and displacement is : $\varepsilon_{ij} = \frac{1}{2}( \frac{\partial u_i}{\partial x_j} + \frac{\partial u_j}{\partial x_i})$\\

The integrability of the strain displacement equations is ensured by the compatibility equations : \\
\begin{equation}
    \begin{gathered}
        \varepsilon_{ipm} \varepsilon_{jqn} \frac{\partial^2 \varepsilon_{mn}}{\partial x_p \partial x_q} = 0\\
        \text{equivalently } \frac{\partial^2 \varepsilon_{ij}}{\partial x_k \partial x_l} + \frac{\partial^2 \varepsilon_{kl}}{\partial x_i \partial x_j} - \frac{\partial^2 \varepsilon_{il}}{\partial x_j \partial x_k} - \frac{\partial^2 \varepsilon_{jk}}{\partial x_i \partial x_l} = 0
    \end{gathered}
\end{equation}

It can be shown that : \begin{itemize}
    \item if strain do not satisfies compatibility then no displacement can be found by integration\\
    \item if these equations are satisfied and the body is simply connected (no holes) then the displacement vector can be found by integration\\
    \item if the solid is not simply connected, any displacement vector may not be unique\\
\end{itemize}

\color{gray}Note : all compatibility equations are second order derivatives, thus if you are given a strain field is proportional to $\underline{x}$, they are automatically satisfied. \color{black}\\

\subsection{Scaling analysis}
This approach is deeply connected to the principle of objectivity - that any physically relevant theory is independent of the reference frame used to describe it.\\
In other words, no matter whether a physical quantity is represented in one set of units or another, its underlying properties exist independent of the units used to measure it.\\
A key consequence of objectivity is the generalized homogeneity\footnote{why we can non-dimensionalize an equation and know it represents the system we wish to describe nonetheless}. This property of physical laws motivates a procedure called dimensional analysis.\\

\textbf{Theory of similitude}. In example : the fundamental equation of motion is valid, irrespective of my local "g"; the gravitational acceleration. \\

\color{gray}Note : M mass, T time, L length are fundamental physical dimension whereas quantities like $\rho$, acceleration are derived quantities. For derived units, we have to appropriately re-scale each quantity involved self-consistently. \color{black}\\

We can usually do an dimensional analysis as follow : $\Pi = \frac{q}{f(p)}$ where $q$ is the parameter we want to approximate and $p$ are the parameters. $\Pi$ is a constant ranging from 0.1 to 10\\

There is a systematic method to make dimensional analysis more robust and useful for model design.\\
This method is codified by the \textbf{Buckingham-$\pi$ theorem}. This theorem relates  : \begin{itemize}
    \item the total \textbf{number of dimensional parameters in our system}, n\\
    \item the \textbf{number of fundamental dimensions}, m\\
\end{itemize}
To the \textbf{number of dimensionless groups that we required to completely describe} our system.\\
\begin{equation}
    \# \text{of dimensionless groups} = n-m
\end{equation}

Suppose we have n parameters $a_1,\dots, a_n$ and a property of interest, $a$.\\
We can write : \begin{equation}
    a = f(a_1,\dots, a_n)
\end{equation}
By writing down the dimensions of each $a_1,\dots, a_n$ we can find $m$.\\
We can also assume that the $m$ independent parameters appear in the first $m$ parameters $a_1, \dots, a_m$. Thus, we can write : \\
\begin{equation}
    \begin{matrix}
        [a] = [a_1]^p\dots [a_m]^r\\
        [a_{m+1}] = [a_1]^{p_{m+1}}\dots [a_m]^{r_{m+1}}\\
        \dots\\
        [a_n] = [a_1]^{p_n} \dots [a_m]^{r_n}\\
    \end{matrix}
\end{equation}

Here, the p's and n's are the exponents of powers of the dimensions of $a_1,\dots, a_m$ that together compose the dimensions of $a_{m+1}\dots a_n$\\

We therefore have the equation : \begin{equation}
    \Pi = \Phi(\Pi_1, \dots, \Pi_{n-m})
\end{equation}
Where each $\Pi$ and $\Phi$ are dimensionless.\\

We form each $\Pi$ by dividing the dimensional group of fundamental parameters that match the dimensions of $a_{m+1},\dots, a_n$ : \begin{equation}
    \Pi = \frac{a}{a_1^p\dots a_m^r} \Rightarrow \Pi_1 = \frac{a_{m+1}}{a_1^{p_{m+1}} \dots a_m^{r_{m+1}}},\quad  \dots , \quad \Pi_{n-m} = \frac{a_{n}}{a_1^{p_{n}} \dots a_m^{r_{n}}}
\end{equation}

\quad \underline{Prototyping :}\\
For a prototype (desired design at scale) we get : \begin{itemize}
    \item prototype : $a^{(p)} = f(a_1^{(p)},\dots, a_n^{(p)})$ a scaled model shares the same function, $\Pi^{(p)} = \Phi(\Pi_1^{(p)}, \dots, \Pi_{n-m}^{(p)})$\\
    \item model : $a^{(M)} = f(a_1^{(M)},\dots, a_n^{(M)})$, $\Pi^{(M)} = \Phi(\Pi_1^{(M)}, \dots, \Pi_{n-m}^{(M)})$\\
\end{itemize}

In order to match the operational conditions of our prototype we must use the same values of $\Pi^{(M)}$ as they occur in our prototype.\\

\begin{equation}
    a^{(p)} = a^{(M)} (\frac{a_1^{(p)}}{a_1^{(M)}})^p \dots (\frac{a_m^{(p)}}{a_m^{(M)}})^r
\end{equation}
A similar relation can be developed for all dimensionless groups : $\Pi_1,\dots, \Pi_{n-m}$\\
\begin{equation}
    \Pi_1^{(M)} = \Pi_2^{(p)} \Rightarrow a_{m+1}^{(M)} = a_{m+1}^{(p)} (\frac{a_1^{(M)}}{a_1^{(p)}})^{p_{m+1}} \dots (\frac{a_m^{(M)}}{a_m^{(p)}})^{r_{m+1}}
\end{equation}
\end{document}