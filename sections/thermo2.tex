\documentclass[../main.tex]{subfiles}
\graphicspath{{\subfix{../IMAGES/}}}

\begin{document}
\localtableofcontents
\subsection{Principes fondamentaux}
\begin{equation}
    \frac{dU_{cz}}{dt} = \sum_k \dot{E^+} + \sum_{\alpha} \dot{Q^+} + \sum_j h_{czj}\dot{M_j^+}
\end{equation}
On définit les énergies totales comme : $U_{cz} = U + M \frac{C^2}{2} + Mgz$\\
Pour un système fermé, on a \\
\begin{equation}
    dU_{cz} = \delta A^+ + \delta Q^+
\end{equation}
On définit aussi l'enthalpie : $H = U + PV$\\
Pour un gaz parfait, on a les relations : \\
\begin{equation}
\begin{split}
    dU = M c_v dT\\
    dH = M c_p dT
\end{split}
\end{equation}
Ces fonctions la sont \textbf{extensives}. \\

On utilise les termes suivants : \\
\begin{table}[hbt!]
    \centering
    \begin{tabular}{c|c|c|c}
    \hline
         & adiabate & Isénerge &Isentropique\\
         \hline
        =0 & $Q^+$ & U & S\\
    \end{tabular}
    \caption{Transformations}
\end{table}

On a aussi pour beaucoup de fonctions : \\
\begin{equation}
    \dot{H_{cz}} = h_{cz} \dot{M}
\end{equation}

On définit l'énergie de transformation : \\
\begin{equation}
        Y^+ = \sum \int (h_{czj}dM_j^+) - \Delta (U_{cz} + P_aV)\\
\end{equation}
\begin{equation}
    \sum \dot{E_k^+} + \sum \dot{Q_i^+} + \dot{Q_a^+} + \sum \dot{Y_n^+} = 0
\end{equation}

\textbf{2ème principe :} L'entropie, pour décrire les évolutions naturelles des systèmes. L'entropie est proportionnel à la fraction d'énergie non convertible en travail. \\

\begin{equation}
    dS = \sum \frac{\partial Q_{\alpha}^+}{T_{\alpha}} + \sum s_j dM_j^+ + \sum \frac{\partial R_{\alpha}}{T_{\alpha}} + \sum (\frac{1}{T_{\beta}}-\frac{1}{T_{\alpha}}) \delta Q_{\alpha\beta} + \sum \partial S_{\alpha
    }^x
\end{equation}

Pour un système ouvert, on a : \\
\begin{equation}
    Du_{cz} = \sum E_k^+ + E_a^+ + \sum Q^+_i + Q_a^+ + \sum h_{czj}dM_{czj}^+
\end{equation}
Où $Q_a$ et $E_a$ sont des échanges avec l'atmosphère.\\

On considère aussi des fluides incompressibles : $\dot{M_j^+} = \frac{\dot{V_j^+}}{v_j} = \frac{C_j S_j}{v_j}$\\

On peut aussi définir les différents coefficients :\\
\begin{table}[hbt!]
    \centering
    \begin{tabular}{||c|c|c|}
        \hline
        Compression isochore& Dilatation isobare & Compressibilité isotherme \\
        \hline
        $\alpha_v = \frac{T}{P} (\frac{\partial P}{\partial T})_v$& $\beta_p = \frac{T}{v} (\frac{\partial v}{\partial T})_P$ & $\gamma_t = -\frac{v}{P} (\frac{\partial P}{\partial v})_T$\\
        \hline
    \end{tabular}
    \caption{Facteurs}
\end{table}

Selon l'équation de Gibbs, on a $dU = -PdV + TdS$\\

\subsubsection{Fermé homogène}
\begin{equation}
    \partial a^- + d\frac{C^2}{2} + gdZ + \partial r = Pdv = -du + \partial q^+ + \partial r = -du + Tds
\end{equation}
Pour un gaz parfait, on a selon Mayer : $c_p-c_v = r$\\
On peut donc définir les valeurs entières : \\
$v_c = v(1+\frac{C^2}{2c_p T})^{\frac{1}{1-\gamma}}$, $P_c = P(1+\frac{C^2}{2c_p T})^{\frac{\gamma}{1-\gamma}}$\\
$T_c = T+\frac{C^2}{2c_p}$, $h_c = h+\frac{C^2}{2}$\\

\begin{table}[hbt!]
    \centering
    \begin{tabular}{c|c}
        Fluides réels & GP-SP \\
          \hline
        $\frac{dT}{T} = \frac{1}{\beta_p} \frac{dv}{v} + \frac{1}{\alpha_v}\frac{dP}{P}$ & $\frac{dT}{T} = \frac{dv}{v} + \frac{dP}{P}$\\
        $du = (\alpha_v - 1) Pdv + c_v dT$ & $du = c_v dT$\\
        $dh = (1-\beta_p)vdP + c_p dT$ & $dh = c_pdT$\\
        $ds = \frac{c_p}{\beta_p} \frac{dv}{v} + \frac{c_v}{\alpha_v}\frac{dP}{P}$ & $ds = c_p \frac{dv}{v} + c_v \frac{dP}{P}$\\
        $ds = \frac{\alpha_v P}{T} dv + c_v \frac{dT}{T}$ & $ds = r \frac{dv}{v}+ c_v \frac{dT}{T}$\\
        $ds = -\frac{\beta_p v}{T}dP + c_p \frac{dT}{T}$ & $ds = -r \frac{dP}{P} + c_p \frac{dT}{T}$\\
        $c_p - c_v = \alpha_v P \frac{dv}{dT} + \beta_p v \frac{dP}{dT}$ & $c_p - c_v = P\frac{dv}{dT}+v\frac{dP}{dT}$\\
        $c_p-c_v = \alpha_v \beta_p \frac{vP}{T}$ & $c_p-c_v = r$\\
    \end{tabular}
    \caption{Formules importantes}
\end{table}

Lors d'une transformation polytrope, on définit le coefficient polytropique : $\sigma = \frac{Tds}{vdP}$\\

Le facteur calorifique : $\Gamma = \frac{\gamma-1}{\gamma} = \frac{r}{c_p}$, avec le rapport calorifique $\gamma = \frac{c_p}{c_v}$\\

\subsubsection{Ouvert homogène}
\begin{equation}
    \partial e^- + d\frac{C^2}{2} + gdZ + \partial r = -vdP = -dh + \partial q^+\partial r = -dh + Tds
\end{equation}
On prend aussi $\dot{M} = \rho \dot{V}$\\
On définit le débit de masse surfacique $\dot{\mu} = \frac{\dot{M}}{S} = \frac{C}{v}$\\


\subsection{Tuyère}
Vitesse du son dans un milieu (vitesse de laval) : \\
\begin{equation}
    C_L = \sqrt{\gamma rT} = Al_l
\end{equation}
On définit donc le nombre de laval : $L_a = \frac{C_1}{A_L}$\\

Dans une tuyère simple on a plusieurs cas :\\
Sonique adapté si : $P_4 = P_L = P_2$. Sinon $P_4 < P_L = P_2$\\

On a les relations suivantes si la tuyère est isentrope :\\
\begin{table}[hbt!]
    \centering
    \begin{tabular}{||c|c|}
    \hline
        C(vitesse) & $\sqrt{2c_p T_0(1-(\frac{P}{P_0})^{\Gamma})} = \sqrt{2c_p (T_0-T)}$ \\
        $M_a$ & $\sqrt{\frac{2}{\gamma -1} ((\frac{P}{P_0})^{\frac{1-\gamma}{\gamma}}-1)}$\\
        $\dot{M} = SP_0 \sqrt{\frac{2\gamma}{(\gamma-1)rT_0}((\frac{P}{P_0})^{\frac{2}{\gamma}}(1-\frac{P}{P_0})^{\frac{\gamma-1}{\gamma}})}$& $\dot{M} = \frac{SC}{v} = \frac{S}{v}\sqrt{2c_p T_0 (1-(\frac{P}{P_0})^{\frac{\gamma-1}{\gamma}})}$\\
        \hline
    \end{tabular}
    \caption{Valeurs}
\end{table}
\subsubsection{Tuyère de Laval}
Au col, on vérifie d'abord que $\frac{P_L}{P_0} > \frac{P_4}{P_0}$. Si c'est vrai, on est sonique au col et $P_2 = P_L$\\

On a plusieurs possibilités : adapté $P_4 = P_3^*$\\
non adapté : $P_3^* > P_4$\\
supersonique adapté : $P_4 = P_3^{**}$\\
supersonique non adapté : $P_4 > P_3^{**}$\\

\subsection{Turbine/compresseur/canal}
\subsubsection{Canal chauffé/refroidi}
\begin{table}[hbt!]
    \centering
    \begin{tabular}{c|c}
        Canal chauffé H & Canal refroidi R  \\
        \hline
        $\eta_{H\sigma} = \frac{q^+}{q^+ + r} = \frac{\ln{\tau}}{\ln{\tau} - \Gamma \ln{\pi}}$ & $\eta_{R\sigma} = \frac{q^- r}{q^-} = \frac{\ln{\tau} - \Gamma \ln{\pi}}{\ln{\tau}}$ \\
        $\eta_{H\sigma} = \frac{h_{cz2} - h_{cz1}}{h_{cz2} - h_{cz1} + r}$ & $\eta_{R\sigma} = \frac{h_{cz1} - h_{cz2} - r}{h_{cz1} - h_{cz2}}$\\
        \hline
        $\eta_{HP} = \frac{h_{cz2} - h_{cz1}}{h_{2P-h_1}} = \frac{\tau -1}{\pi -1}$ & $\eta_{RP} = \frac{h_1-h_{2P}}{h_{cz1} - h_{cz2}} = \frac{\frac{\tau}{\pi^{\Gamma}}-1}{\tau-1}$\\
        \hline
        $\eta_{HT} = \frac{h_{cz2} - h_{cz1}}{T_1(s_2-s_1)} = \frac{\tau-1}{\ln{\tau}-\Gamma \ln{\pi}}$ & $\eta_{RT} = \frac{T_1(s_1-s_2)}{h_{cz1} - h_{cz2}} = \frac{\ln{\tau}-\Gamma \ln{\pi}}{\tau -1}$
    \end{tabular}
    \caption{Rendements}
\end{table}

Avec $\tau = \frac{T_2}{T_1}$, $\pi = \frac{P_2}{P_1}$\\

\subsubsection{Turbine/compresseur}
On a en général : $\dot{E^-} = \dot{M}e^- = \dot{M}(h_1-h_4 + q^+)$\\
\begin{table}[hbt!]
    \centering
    \begin{tabular}{c|c}
        Turbine & Compresseur \\
        \hline
        $\eta_{T\sigma} = \frac{e^-}{-\int_1^4 vdP - \frac{\Delta C^2}{2}}$ & $\eta_{C\sigma} = \frac{\int_1^4 vdP + \frac{\Delta C^2}{2}}{e^+}$\\
        $\eta_{Ts} = \frac{e^-}{-\int_1^{4s} vdP - \frac{\Delta C^2}{2}}$ & $\eta_{Cs} = \frac{\int_1^{4s} vdP + \frac{\Delta C^2}{2}}{e^+}$\\
        $\eta_{TT} = \frac{e^-}{-\int_1^{4T} vdP - \frac{\Delta C^2}{2}}$ & $\eta_{CT} = \frac{\int_1^{4T} vdP + \frac{\Delta C^2}{2}}{e^+}$\\
    \end{tabular}
    \caption{Rendements réels}
\end{table}
Pour des gaz parfaits :\\
\begin{table}[hbt!]
    \centering
    \begin{tabular}{c|c}
        Turbine & Compresseur \\
        \hline
        $\eta_{T\sigma} = \frac{-\ln{\tau}}{-\Gamma \ln{\pi}}$& $\eta_{C\sigma} = \frac{\Gamma \ln{\pi}}{\ln{\tau}}$\\
        \hline
        $\eta_{Ts} = \frac{1-\tau}{1-\pi^{\Gamma}}$ & $\eta_{Cs} = \frac{1-\pi^{\Gamma}}{1-\tau}$\\
        \hline
        $\eta_{TT} = \frac{1-\tau}{-\Gamma \ln{\pi}}$ & $\eta_{CT} = \frac{\Gamma \ln{\pi}}{\tau-1}$\\
    \end{tabular}
    \caption{Rendement Gaz parfait}
\end{table}
En gaz parfait, on a les relations : \\
\begin{table}[hbt!]
    \centering
    \begin{tabular}{c|c|c|c}
        \hline
         & Monoatomique & Biatomique & Triatomique et plus \\
        $c_v$ & $3\frac{r}{2}$ & $5\frac{r}{2}$ & $3r$\\ 
        $c_p$ & $5\frac{r}{2}$ & $7\frac{r}{2}$ & $4r$\\
    \end{tabular}
    \caption{Valeurs pour $c_p$ et $c_v$}
\end{table}

\subsection{Gaz parfait/réels}
De manière générale : \\
\begin{equation}
    Pv = rT
\end{equation}
Une constante : $\Tilde{r} = r\Tilde{m} [\frac{J}{K kmol}] = 8314$\\
Pour des gaz semi-parfait, les coefficients calorifiques dépendent de la température.\\
Volume molaire : $\Tilde{v} = \frac{V}{N_x}$\\

L'état normal est définit pour $P= 1atm = 1bar = 10^5Pa$ et $T_0 = 0^{\circ}C$\\
Le volume normal est donc de $\Tilde{v_n} = \frac{\Tilde{r} T_n}{P_n} = 22.4 [\frac{m^3}{kmol}]$ et un mètre cube normal $1Nm^3 = \frac{1}{22.4}[kmol]$\\

\quad \underline{Vander-Walls :}\\
\begin{equation}
    (v-b)(P+\frac{a}{v^2})=rT
\end{equation}

Pour les solides et liquides, on q $c_p \simeq c_v \simeq c$\\
$du = cdT$, $dh=cdT+vdP$, $ds = \frac{c}{T}dT$\\

\subsubsection{Changement de phase}
\warning Ici les pressions sont en bar et les températures en degré celcius\\
Par clapeyron :\\
\begin{equation}
    \frac{dP_s}{dT} = \frac{s"-s'}{v"-v'} = \frac{q_{vap}}{(v"-v')T} > 0
\end{equation}
Pour trouver le point de saturation :\\
\begin{equation}
    \ln{(\frac{P_v"}{140974})} = \frac{-392805}{231.667+\hat{T}}
\end{equation}

On a le titre de mélange : $x = \frac{M"}{M}$\\
Pour toute fonction lors d'un changement de phase, on a \\
\begin{equation}
    \overline{v} = v' + x(v"-v')
\end{equation}

\subsubsection{Mélange de gaz parfait}
Concentration massique : $c_i = \frac{M_i}{M}$ et molaire $\Tilde{c_i} = \frac{N_i}{N}$\\
Avec les relations : $M = \sum M_i$ et $\sum c_i = \sum \Tilde{c_i} = 1$, $N = \sum N_i$\\
\begin{equation}
    \Tilde{m_i} \Tilde{c_i} = \Tilde{m} c_i
\end{equation}
De plus, $\Tilde{m_i} = \frac{M_i}{N_i}$ et $\frac{P_i}{P} = \Tilde{c_i}$\\

Ainsi que $r= \sum c_ir_i$, $c_v = \sum c_i c_{vi}$, $c_p = \sum c_i c_{pi}$\\

Selon Dalton, on a le volume constant et $P= \sum P_i$\\
Selon Amagat, on a la pression constante et $V = \sum V_i$\\
Entropie de diffusion : $S^d = -N \Tilde{r} \sum \Tilde{c_i} \ln{\Tilde{c_i}}$\\
Température finale :\\

$T = \frac{\sum \Tilde{c_i} \Tilde{c_{vi}} T_{i1}}{\sum \Tilde{c_i} \Tilde{c_{pi}}}$ De même avec $c_p$. \\
La pression finale : $P = \frac{N\Tilde{r}T}{V} = \frac{T}{V} \sum \frac{V_{1i} P_{1i}}{T_{i1}}$\\

\subsection{Saturation}
On définit le degré de saturation : $\psi = \frac{w^*}{w^{*"}} = \frac{M_e}{M_v"}$\\
Taux d'humidité : $w^* = \frac{\Tilde{m_v}}{\Tilde{m_a}} \frac{P_v}{P_m-P_v}$\\
Ainsi que l'humidité relative : $\varphi = \frac{P_v}{P_v"}$\\

On a $P_m = P_a+P_v$ ainsi que $M = M_e+M_a$\\

Pour un mélange saturé, on a $\hat{h^*"} = c_{pa} \hat{T} + w^*"(q_{vap} + c_{pv} \hat{T})$\\
Pour un mélange vapeur, liquide : $\hat{h^*} = \hat{h^*"} + (w^*-w^*")c_l \hat{T}$\\
Pour un mélange sec : $\hat{h^*} = c_{pa}\hat{T} + w^*(q_{vap} + c_{pv}\hat{T})$\\

On a aussi $r = \frac{r_a + w^*r_v}{1+w^*}$\\

\begin{equation}
\begin{split}
        w^* = \frac{\Tilde{m_v}}{\Tilde{m_a}} \frac{\varphi P_v"}{P_m - \varphi P_v"}\\
        \Rightarrow w^* = \frac{\dot{M_e}}{\dot{M_a}} = \frac{\dot{M_{a1}} w_1^* + \dot{M_{a2}} w_2^*}{\dot{M_{a1}} + \dot{M_{a2}}}
\end{split}
\end{equation}

De même pour $\hat{h^*}$ : \\
$\hat{h^*} = \frac{\dot{H}}{\dot{M_a}} = \frac{\dot{M_{a1}} \hat{h_1^*} + \dot{M_{a2}} \hat{h_2^*}}{\dot{M_{a1} + \dot{M_{a2}}}}$\\

Si on a mélange avec une substance pure E :\\
$w_2^* = \infty$, $\dot{M_{a2}} = 0 \Rightarrow w^* = w_1^* + \frac{\dot{M_{e2}}}{\dot{M_{a1}}}$\\

\subsection{Exergie}

On définit ici la partie de l'énergie qui peut réellement être utilisée.\\
On a donc le facteur de carnot $\theta = 1-\frac{T_a}{T}$\\
Exergie chaleur : $E_q^+ = \int \theta \partial Q^+$\\
On a donc la coénergie totale : $J_{cz} = U_{cz} + P_aV - T_aS$\\
Ainsi que la coenthalpie totale : $K_{cz} = H_{cz} - T_aS$\\

L'exergie transformation reçu est donc $\dot{E_y} = \sum (h_{czj} \dot{M_j^+}) - \frac{dJ_{czj}}{dt}$\\

On définit les pertes comme : \\
\begin{equation}
    \sum \dot{E^+_k} + \sum \dot{E^+_{qi}} + \sum \dot{E^+_{yn}} = \dot{L} \geq 0
\end{equation}
Et donc le rendement exergétique : \warning Faire attention aux bordures!\\

\begin{equation}
    \eta = \frac{\sum \dot{E^-_k} + \sum \dot{E^-_{qi}} + \sum \dot{E^-_{yn}}}{\sum \dot{E^+_k} + \sum \dot{E^+_{qi}} + \sum \dot{E^+_{yn}}}
\end{equation}

\subsubsection{Turbine}
$\eta = \frac{\sum h_{cj} \dot{M_j^+} - \dot{Q_a}^-}{\sum k_{cj} \dot{M_j^-}} = 1-\frac{T_a (s_2-s_1)}{k_{1c}-k_{2c}}$\\
$\dot{L} = \dot{Q_a^-} - T_a \sum \dot{M^+_j}s_j$\\


\subsubsection{Compresseur}
$\eta = \frac{\sum k_{cj} \dot{M_j^-}}{\sum h_{cj} \dot{M_j^+}} = 1-\frac{T_a (s_2-s_1)}{h_{1c}-h_{2c}}$\\
$\dot{L} =  - T_a \sum \dot{M^+_j}s_j$\\

\subsection{Combustion}
On définit le pouvoir énergétique isobare :\\
massique : $\underline{\Delta}h^{\circ} = \frac{H_M^{\circ}-H_{Gc}^{\circ}}{M_f}[\frac{J}{kmol_{fuel}}]$\\
molaire : $\underline{\Delta}\Tilde{h^{\circ}} = \frac{H_M^{\circ}-H_{Gc}^{\circ}}{N_f}[\frac{J}{kmol_{fuel}}]$\\

On a maintenant $M_{Gc} = M_M = M_F+M_A$\\
On prend le pouvoir exergétique supérieur si on trouve de l'eau sous forme liquide dans les produit de combustion. Sinon on prend le pouvoir exergétique inférieur.\\

L'énergie de combustion est donnée par $\dot{Y}_{comb}^+ = \dot{M_f} \underline{\Delta}h_s^{\circ}$\\

Soit $\underline{\Delta}h_i^{\circ} = \underline{\Delta}h_s^{\circ} - \dot{M_{h_2O}} q_{vap}$\\
L'exergie de combustion est donc : $\dot{E}_{ycomb}^+ = \dot{M}_f \underline{\Delta}k^{\circ}$\\
\end{document}