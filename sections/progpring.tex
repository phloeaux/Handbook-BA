\documentclass[../main.tex]{subfiles}
\graphicspath{{\subfix{../IMAGES/}}}

\begin{document}
\localtableofcontents
\subsection{Matlab}
\subsubsection{Array/matrix}
Indexes start at 1\\
Line vector : $[x_1 x_2 \dots x_n]$(or with ,in between);
Column vector : $[x_1; x_2; \dots ; x_n]$;
Accessing array element : X(n);
Create array with start and end values : $X=start:pas:end$\\
Create evenly spaced values in array : linspace(start, end, number of values)\\
Accessing multiple element at once : $X([x_n x_m \dots])$\\

\begin{minipage}{.5\textwidth}
\begin{itemize}
    \item longueur : length(X);\\
    \item somme : sum(X);\\
    \item moyenne : mean(X);\\
    \item trier : sort(X), if matrix, we can sort by rows sortrows(X);\\
\end{itemize}
\end{minipage}
\vline
\begin{minipage}{.5\textwidth}
    We can also have a polynom declared as an array :\\
    $p=[1, -6, -72, -27]$ which means $x^3 - 6x^2 -72x - 27$\\
    To find roots, r=roots(p)\\
    We can find back the polynom from roots $poly(r)$\\
\end{minipage}
In an array we have the following :$\begin{pmatrix}
1 & 4 & 7\\
2 & 5 & 8\\
3 & 6 & 9\\
\end{pmatrix}$
To delete a case : X(position) = []\\

Create a matrix filled with 0 : zeros(dimension)\\
Filled with 1 : ones(dimension)\\
Identity matrix : eye(dimension)\\
Get diagonal elements : diag(matrix)\\

Create a string with numbers and string : X = ['phrase 1'+value 'phrase 2'+string(number if it doesn't work the other way)]\\

Get indexes of special element in array : find(condition, number of element, which elements (ie 'last'))\\

If we need multiple types of values in an array : myCell={$\dots$; $\dots$};\\
To change a cell in array : cell2mat(myCell);\\
(inverse : mat2cell)\\




\subsubsection{Operations}
Transpose : X'\\
If the matrix contains imaginary number, X' will transpose the matrix and take the conjugate of each number. If you only want the transpose of the matrix : X.'\\
If you want to work on each value of a matrix then add a point before any operation :\\
\begin{itemize}
    \item multiply : X.*Y\\
    \item division : X./Y\\
    \item exponant : X.\^{} 3\\
\end{itemize}
Otherwise, it will do a basic matrix operation.\\

To get a timer : tic to start it\\
toc to finish and get the time difference.\\

\subsubsection{Function}
syntax : \\
function out1 = funcname(in1);\\
end

\warning name file has to be the same as function name\\

Define global variable : global c\\

if cond1\\
\quad command\\
elseif cond2\\
\quad command\\
end\\

for n=1:0.5:10 (n prend les valeurs du tableau créé)\\
\quad command\\
end\\

\quad \underline{Anonymous function :}\\
name = @(param) core;\\
to plot them : ezplot(eq, [x1, x2]); (x1, x2 can be omitted)\\

\quad \underline{Integral}
to integrate : integral(function, x1, x2);\\
trapz(x,y);(with x the step)\\
symbolic integral : int(sin(x));\\



\subsubsection{Plot}
plot(x,y,options)\\

if multiple plots : \\
hold on;\\
plot(x1, y1, '-b'); (in blue with line)\\
plot(x2,y2, 'color', '\#colornumber', 'LineStyle', '-');(same)\\
hold off;\\

to let multiple plot appear in different windows : figure\\
if we need to modify a plot afterwards, use the set(namefig, 'which data', new data)\\

3D plot :[x y]= meshgrid(minval:maxval) (create a 2D grid for both X and Y to increase speed of calcul)\\
mesh : points are linked, stem : only points, contour : creates line between surfaces, surf: creates a surface between points\\

\subsubsection{Interpolation}
find outliers : isoutlier(tab, 'movmedian', numvalues);\\
numvalues is the number of values over which we are doing the median.\\

to interpolate : interp1(x, y, newpoint, 'linear');\\
newpoint should be an array containing the position where we need to inteprolate\\

\subsubsection{Polynômes}
find roots : roots(p);\\
find polynom from roots : poly(r);\\
evaluate polynom defined by x : polyval(p,x);(gives us the value of p at points x)\\

adjust a polynom to some values : polyfit(p,x, order);\\

\subsection{C}
\subsubsection{Memory allocation}
int *tab= malloc(size*size element);\\
free(tab);(memory is freed not deleted)\\
(calloc does the same but initialise the array to zero)\\

\subsubsection{Numbers}
At the beginning of the file we need to include : $\#$include <limits.h> in order to be able to do bit wise operations\\

\quad \underline{Bit wise operations :}\\
\begin{itemize}
    \item and : $\&$\\
    \item or : |\\
    \item xor : $\hat{}$ \\

    \item 0x00115533 $\&$ 0x00FF0000 = 0x00110000\\
    \item 0x00110000>>16 = 11\\
\end{itemize}

\subsubsection{Files}
int ferror(FILE *fp); (is there any error?)\\
int perror(const char *s); (print string s and gives a description of an error)\\



\end{document}